\section{(Materials and) Methods}

Experimental data (work in progress)

Analysis methods: GLM (logistic model) to discern between treatment and covariate effect. 
Overdispersion control.

Multiple new features were integrated into the methylKit R-package: The ability to take into account covariates and overdispersion when calculating differential methylation and the possibility to simulate methylation data for testing and demonstration purposes.

The first new feature - or, more precisely, the first two features - aim to  improve the sensitivity of the detection of differentially methylated sites by canceling out sites where the difference is either the result of a covariate and not the primary treatment as well as those sites where differences occur as a result of statistical overdispersion.

Covariates must be supplied by the user and theoretically could encompass metadata such as age, sex, etc. which are generally thought to influence DNA methylation quite significantly [QUOTE] . It follows, that the ability to estimate / eliminiate their influence on the calculation of differentially methylated sites (caused by a certain treatment) would be of great value.

This is achieved using a GLM (more precisely, a logistic model): A model ist constructed with the covariates as well as the treatment information as model covariates, which is then used to calculate ...

\\
Data Simulation methods: Binomial distribution simulates background methylation. 
Betabinomial distribution maps this to actual percentages.
Logistic function to simulate influence of numeric covariate (i.e. age) on methylation percentages for individual CpG sites.
  
  